\documentclass[sigconf]{acmart}

\settopmatter{printacmref=false} % Removes citation information below abstract
\renewcommand\footnotetextcopyrightpermission[1]{} % removes footnote with conference information in first column
\pagestyle{plain} % removes running headers

\renewcommand{\baselinestretch}{0.99}


\usepackage{booktabs} % For formal tables
\usepackage{verbatim}
\usepackage{fancynum}
\usepackage{multirow}
\usepackage{graphicx}
\usepackage{bm}
%\usepackage{amsthm}
%\usepackage{floatrow}
%\usepackage{subfigure}

% Copyright
%\setcopyright{none}
%\setcopyright{acmcopyright}
%\setcopyright{acmlicensed}
\setcopyright{rightsretained}
%\setcopyright{usgov}
%\setcopyright{usgovmixed}
%\setcopyright{cagov}
%\setcopyright{cagovmixed}

\begin{comment}
% DOI
\acmDOI{10.475/123_4}

% ISBN
\acmISBN{123-4567-24-567/08/06}

%Conference
\acmConference[WOODSTOCK'97]{ACM Woodstock conference}{July 1997}{El
  Paso, Texas USA}
\acmYear{1997}
\copyrightyear{2016}

\acmPrice{15.00}

\acmSubmissionID{123-A12-B3}
\end{comment}

\begin{document}
\title{Knowledge Graph Embedding with Triple Context}

\author{ }
%\authornote{}
\orcid{ }
\affiliation{%
  \institution{ }
  \streetaddress{ }
  \city{ }
  \state{ }
  \postcode{ }
}
\email{ }


% The default list of authors is too long for headers}
\renewcommand{\shortauthors}{}

\begin{abstract}
Knowledge graph embedding, which aims to represent entities and relations in low dimensional vector spaces, has shown outstanding performance on a few knowledge graph completion tasks, such as link prediction and triple classification. Most existing methods are based on the assumption that a knowledge graph is a set of separate triples, ignoring rich graph features, i.e., structural information in the graph. In this paper, we take advantages of structures in knowledge graphs, especially local structures around a triple, which we refer to as \textit{triple context}. We then propose a Triple-Context-based knowledge Embedding model (TCE). For each triple, two kinds of structure information are considered as its context in the graph; one is the outgoing relations and neighboring entities of an entity and the other is relation paths between a pair of entities, both of which reflect various aspects of the triple. Triples along with their contexts are represented in a unified framework, in which way structural information in triple contexts can be embodied. The experimental results show that our model outperforms the state-of-the-art methods for link prediction. The source code of this paper is available online\footnote{\url{https://goo.gl/EflFHL}}.
\end{abstract}

%
% The code below should be generated by the tool at
% http://dl.acm.org/ccs.cfm
% Please copy and paste the code instead of the example below.
%
\begin{CCSXML}
<ccs2012>
 <concept>
  <concept_id>10010520.10010553.10010562</concept_id>
  <concept_desc>Computer systems organization~Embedded systems</concept_desc>
  <concept_significance>500</concept_significance>
 </concept>
 <concept>
  <concept_id>10010520.10010575.10010755</concept_id>
  <concept_desc>Computer systems organization~Redundancy</concept_desc>
  <concept_significance>300</concept_significance>
 </concept>
 <concept>
  <concept_id>10010520.10010553.10010554</concept_id>
  <concept_desc>Computer systems organization~Robotics</concept_desc>
  <concept_significance>100</concept_significance>
 </concept>
 <concept>
  <concept_id>10003033.10003083.10003095</concept_id>
  <concept_desc>Networks~Network reliability</concept_desc>
  <concept_significance>100</concept_significance>
 </concept>
</ccs2012>
\end{CCSXML}

\begin{comment}
\ccsdesc[500]{Computer systems organization~Embedded systems}
\ccsdesc[300]{Computer systems organization~Redundancy}
\ccsdesc{Computer systems organization~Robotics}
\ccsdesc[100]{Networks~Network reliability}
\end{comment}

\keywords{Knowledge Graph, Representation Learning, Triple Context}


\maketitle

\section{Introduction}
Knowledge graphs (KGs) are composed of structured information of the real world. In the mean while, KGs is becoming a more crucial resource for many AI applications such as QA system and semantic search in Web. Although typical KGs such as NELL, Freebase and DBpeida are large in size, usually containing thousands of relation types, millions of entities and billions of facts, they are far from complete and an incomplete KG cannot effectively support the KG-based applications. Recently, with the development of Internet, a large number of documents contains thousands of relational types data are easily obtainable. These documents is one of the most important resource to automatically extract facts for supplying KGs. Naturally, there is an increasing interest in the relation extraction task.

Despite the success of previous supervised approaches in relation extraction, most of these approaches need sufficient labelling data for training the model. In fact, it is very difficult for manual annotation the training data which is a huge time consuming and labor incentive task. Because of lacking of training data, supervised approaches have a huge challenge in large scale relation extraction. In order to overcome this problem, [1] proposed the distance supervision approach which helps automatically generate new training data by taking an intersection between a text corpus and knowledge base. The distant supervision assumption states that for a pair of entities participating in a relation, any sentence mentioning that entity pair in the text corpora is a positive example for the relation fact. However, the assumption of the distance supervision is too strong and may introduce noise such as false negative samples due to missing facts in the KG. Since neural network approaches have been verify to reduce the effect of noisy data [] [] [].In order to improve the effective for classifying relations from plain text, [] considers finer-grained information,and achieves the state-of-the-art performance.

Most of the existing distance supervision approaches only learn from those sentences contain both two target entities. However, these approaches suffer some drawbacks. Distance supervision approaches mainly focus on how to reduce the noisy data of the automatically manually annotation training data. This kind of approaches can not provide enough information to coverage the various semantic representations. Thus, the performance of distance supervision approaches are not very well. In order to gain more information. [] proposed a novel approach which utilizes the inference chains to improve the previous approach. Although, this approach have only been partial successful, lacking of the prior knowledge will lead to some improper inference chains are selected. For example, if we know h is father of t, the selected inference chain could be e is friend of t and h is a friend of e. In our common sense knowledge, the selected chain row cannot help to infer the relation father-of between h and t.

In this paper, as illustrate in Fig.1, we introduce a KG as the common sense knowledge for this task and propose a multimodal path based relation extraction model. First we employ a neural network model to embed semantic sentences which are selected by a pair of entities. Afterward, we select the most proper path by a path encoder and a pre training PTransE. The KG in this procedure can reduce the noisy inference chains. Finally, we combine sentences, path and the KG as a graph to extract relations.

We have conducted preliminary experiment on a benchmark data set and assessed our method on relation extraction task by convention. The experimental results show impressive improvements on predictive accuracy compared to other baselines.

\section{Related Work}
Distant supervision for RE is originally proposed in (Craven et al., 1999). They focus on extracting binary relations between proteins using a protein KB as the source of distant supervision. Afterward, (Mintz et al., 2009) aligns plain text with Freebase, by using distant supervision . However, most of these methods heuristically transform distant supervision to traditional supervised learning, by regarding it as a single-instance single-label problem,while in reality,one instance could correspond with multiple labels in different scenarios and vice versa. To alleviate the is sue, (Riedel et al., 2010) regards each sentence as a training instance and allows multiple instances to share the same label but disallows more than one label. Further, (Hoffmann et al., 2011; Surdeanu et al., 2012) adopt multi-instance multilabel learning in relation extraction. The main drawback of these methods is that they obtain most features directly from NLP tools with inevitable errors, and these errors will propagate to the relation extraction system and limit the performance

Recently, deep learning (Bengio, 2009) has been successfully applied in various areas, including computer vision,speech recognition and soon. Meanwhile,its effectiveness has also been verified in many NLP tasks such as sentiment analysis(dos Santos and Gatti, 2014), parsing (Socher et al., 2013), summarization (Rush et al., 2015) and machine translation (Sutskever et al., 2014). With the advances of deep learning, there are growing works that design neural networks for relation extraction. (Socher et al., 2012) uses a recursive neural network in relation extraction,and(Xu et al., 2015; Miwa and Bansal, 2016) further use LSTM.(Zengetal.,2014;dosSantosetal.,2015) adopt CNN in this task, and (Zeng et al., 2015; Lin et al., 2016) combine attention-based multi-instance learning which shows promising results. However, these above models merely learn from those sentences which directly contain both two target entities. The important information of those relation paths hidden in the text is ignored. In this paper, we propose a novel path-based neural RE model to address this issue. Besides, although we choose CNN to test the effectiveness of our model, other neural models could also be easily adapted to our architecture.

Relation paths have been taken into consideration on large-scale KBs for relation inference. Path Ranking algorithm (PRA) (Lao and Cohen, 2010) has been adopted for expert finding (Lao and Cohen,2010),information retrieval(Lao et al., 2012), and further for relation classification based on KB structure (Lao et al., 2011; Gardner et al., 2013). (Neelakantan et al., 2015; Lin et al., 2015; Das et al., 2016; Wu et al., 2016) use recurrent neural networks (RNN) to represent relation paths based on all involved relations in KBs.(Guu et al., 2015) proposes an embedding-based compositional training method to connect the triple knowledge for KB completion. Different from the above work of modeling relation paths in KBs,our model aims to utilize relation paths in text corpus, and help to extract knowledge directly from plain text.

\subsection{Neighbor Context}
Neighbor context of an entity is the surroundings of it in KG. It is the local structure that interacts most with the entity and can reflect various aspects of the entity. Specifically, given an entity $e$, the neighbor context of $e$ is a set $C_N(e)=\{(r,t)|\forall r, t, (e,r,t)\in\mathcal{K}\}$, where $r$ is an outgoing edge (relation) from $e$ and $t$ is the entity it reaches through $r$. In other words, the neighbor context of $e$ is all the \textit{relation-tail} pairs appearing in triples with $e$ as the head. For example, as shown in Figure~\ref{pic1}, the neighbor context of entity $h$ is $C_N(h)=\{(r_4, e_1), (r_3, e_2), (r_2, e_3), (r_1, e_8), (r_1, e_{10})\}$. We predict the appearance of an entity based on its neighbor context in our model, as a measurement of the compatibility of the entity and its neighbor context.

\begin{figure}
  \includegraphics[width=0.45\textwidth]{pic1.png}
  \caption{An illustration of the \emph{triple context} of a triple $(h,r,t)$ in a knowledge graph.}
  \label{pic1}
\end{figure}


\subsection{Path Context}
Path context of a pair of entities is the set of paths that starts from an entity to the other in a KG. It is helpful in modeling the relation and capturing interactions between the pair of entities. Given a pair of entities $(h,t)$, the path context of $(h,t)$ is a set $C_P(h,t)=\{p_i | \forall r_{m_1}, \cdots, r_{m_i}, e_1, \cdots, e_{m_i-1},$ $p_i=(r_{m_1}, \cdots, r_{m_i}), (h,r_{m_1},e_1)\in\mathcal{K}, \cdots, (e_{m_i-1}, r_{m_i}, t)\in\mathcal{K}\}$, where $p_i=$ is a list of relations (labeled edges) through which it can traverse from $h$ to $t$, $m_i$ is the length of path $p_i$. In Figure~\ref{pic1}, the path context between $h$ and $t$ is $C_P(h,t) = \{(r_1, r_2), (r_2, r_1, r_2)\}$. We use the path context to predict the tail entity of a triple given the head entity.


\subsection{Triple-Context-based Model}
\begin{comment}
In general, a KG embedding model defines a score function for evaluating the confidence of a triple, and optimize an objective function constructed from the score function. For example, TransE wants $\bm{\mathrm{h}} + \bm{\mathrm{r}} \approx \bm{\mathrm{t}}$ when $(h,r,t)$ holds, so it defines the score function as
\begin{equation}\label{TransE_score_function}
  f(h,r,t)=\|\bm{\mathrm{h}} + \bm{\mathrm{r}} -\bm{\mathrm{t}}\|
\end{equation}
which is negative correlation with the confidence of a triple. It could be either the $L_1$ or the $L_2$ norm.\end{comment}
So far, we have introduced neighbor context and path context, based on which we can define triple context. The triple context of triple $(h,r,t)$ is composed of the neighbor context of the head entity $h$ and the path context of the entity pair $(h,t)$, which can be formalized as:
\begin{equation}\label{triple context}
  C(h,r,t) = C_N(h) \cup C_P(h, t)
\end{equation}

The triple context of a triple can be considered to embody the surrounding structures of it in the graph, which makes the model aware of the information contained in graph structures.

In general KG embedding models, the score function of a triple is only related to the embeddings of entities and relations. For example, TransE defines the score function as $f_{TransE}(h,r,t)=\|\bm{\mathrm{h}} + \bm{\mathrm{r}} -\bm{\mathrm{t}}\|_{L_1/L_2}$. In our method, triple context is introduced in the score function. Given a candidate triple $(h,r,t)$, the score function is the conditional probability that the triple holds given the triple context and all the embeddings, as follows:
\begin{equation}\label{score_function}
  f(h,r,t) = P((h,r,t)|C(h,r,t);\Theta)
\end{equation}
where $C(h,r,t)$ is the triple context of $(h,r,t)$. A higher score of a triple indicates that it holds to a greater extent.

We define an objective function by maximizing the joint probability of all triples in knowledge graph $\mathcal{K}$, which can be formulated as:
\begin{align} \label{joint_prob}
  P(\mathcal{K}|\Theta) &= \prod_{(h,r,t)\in \mathcal{K}} f(h,r,t)
\end{align}

For the score function in Eq.~\eqref{score_function}, we use conditional probability formula to decompose the probability $P((h,r,t)|C(h,r,t);\Theta)$ as:
\begin{align} \label{decomposition}
  \begin{split}
    f(h,r,t) &= P(h|C(h,r,t);\Theta) \\
    & \cdot P(t|C(h,r,t),h;\Theta) \\
    & \cdot P(r|C(h,r,t),h,t;\Theta)
  \end{split}
\end{align}
where the evaluation of the whole triple is decomposed into three parts. The probabilities that $h$, $t$ and $r$ appear given respective condition are determined in turn in these three parts.
% $h$ is determined first based on triple context, then $t$ is determined based on $h$ and triple context, and finally $r$ likewise.

The first part $P(h|C(h,r,t);\Theta)$ in Eq.~\eqref{decomposition} represents the conditional probability that $h$ is the head entity given the triple context and all the embeddings. Since whether $h$ appears is decided mostly by the neighboring structures of $h$ in the KG, we can approximate $P(h|C(h,r,t);\Theta)$ as $P(h|C_N(h);\Theta)$, where $C_N(h)$ is the \emph{neighbor context} of $h$ in the KG. The approximated probability $P(h|C_N(h);\Theta)$ can be considered as the compatibility between $h$ and its neighbor context, it is formalized as a softmax-like representation, which is also used in~\cite{DBLP:conf/emnlp/WangZFC14} and has been validated, as follows:
\begin{align} \label{P_h}
  P(h|C_N(h);\Theta) = \frac{\exp(f_1(h, C_N(h)))}{\sum_{h' \in \mathcal{E}} \exp(f_1(h', C_N(h)))}
\end{align}
where $f_1(\cdot, \cdot)$ is the function that describes the correlation between an arbitrary entity $h'$ and entity context of the specific entity $h$. $f_1(h',C_N(h))$ is defined as:
\begin{equation}\label{f1}
  f_1(h',C_N(h)) =-\frac{1}{|C_N(h)|}\sum_{(r,t)\in C_N(h)} \|\bm{\mathrm{h'}}+ \bm{\mathrm{r}} - \bm{\mathrm{t}}\|_{L_1/L_2}
\end{equation}
where $|C_N(h)|$ is the size of neighbor context $C_N(h)$. Given an entity $h'$ and a pair $(r,t)$ from $C_N(h)$, Eq.~\eqref{f1} uses the score function of TransE to measure the relevance between them, and then take the negative average of all conditions, thus the relevance between $h'$ and $C_N(h)$ is reflected.

The second part $P(t|C(h,r,t),h;\Theta)$ in Eq.~\eqref{decomposition} is the conditional probability that $t$ is the tail entity given the head entity $h$, triple context and all the embeddings. It is the probability that $t$ could be related to $h$ through a potential relation, which can be considered as the relatedness between $h$ and $t$. We use path context between $h$ and $t$ to measure the relatedness of them and approximate $P(t|C(h,r,t),h;\Theta)$ as $P(t|C_P(h,t),h;\Theta)$, where $C_P(h,t)$ is the \emph{path context} between $h$ and $t$. The approximated probability $P(t|C_P(h,t),$ $h;\Theta)$ is formalized as follows:
\begin{align}\label{P_t}
  P(t|C_P(h,t),h;\Theta) = \frac{\exp(f_2(t,C_P(h,t)))}{\sum_{t' \in \mathcal{E}} \exp(f_2(t', C_P(h,t)))}
\end{align}
where $f_2(\cdot, \cdot)$ is a function of correlation between an arbitrary entity $t'$ and path context of the specific entity pair $(h,t)$, which is formalized as:
\begin{equation}\label{f2}
  f_2(t',C_P(h,t)) = -\frac{1}{|C_P(h,t)|}\sum_{p\in C_P(h,t)} \|\bm{\mathrm{h}} + \bm{\mathrm{p}} - \bm{\mathrm{t'}}\|_{L_1/L_2}
\end{equation}
where $\bm{\mathrm{p}}$ composes all relations in $p$ into a single vector by summing over all their embeddings. For example, for path $p_i=(r_{m_1}, \cdots, r_{m_i})$, the embedding of it is $\bm{\mathrm{p}}_i = \bm{\mathrm{r}}_{m_1} + \cdots + \bm{\mathrm{r}}_{m_i}$. Eq.~\eqref{f2} has a similar meaning with Eq.~\eqref{f1} and it reflects the relevance between $t'$ and $C_P(h,t)$.

The last part $P(r|C(h,r,t),h,t;\Theta)$ in Eq.~\eqref{decomposition} is the conditional probability describing that relation $r$ holds given the triple context, head entity $h$, tail entity $t$ and all the embeddings. Since $h$ and $t$ have been determined and triple context has been incorporated in the previous two parts, we can omit $C(h,r,t)$ in $P(r|C(h,r,t),h,t;\Theta)$, which is then formulated as follows:
\begin{align}\label{P_r}
  P(r|h,t;\Theta) = \frac{\exp(f_3(h,r,t))}{\sum_{r' \in \mathcal{R}} \exp(f_3(h,r',t))}
\end{align}
where $f_3(\cdot, \cdot, \cdot)$ is a function indicating the connection between relation $r$ and entity pair $(h,t)$. Here, we use the assumption that $\bm{\mathrm{h}} + \bm{\mathrm{r}} \approx \bm{\mathrm{t}}$ in TransE~\cite{BordesUGWY13} and define function $f_3(h,r,t)$ as:
\begin{align}\label{f3}
  f_3(h,r,t) = -\|\bm{\mathrm{h}} + \bm{\mathrm{r}} - \bm{\mathrm{t}}\|_{L_1/L_2}
\end{align}
where the minus sign is used to make sure that the value of function $f_3(h,r,t)$ is positive correlation with the confidence of triple $(h,r,t)$.

Thus, Eq.~\eqref{decomposition} can be approximated as:
\begin{equation}\label{decomposition_approx}
  f(h,r,t) \approx P(h|C_N(h);\Theta) \cdot P(t|C_P(h,t),h;\Theta) \cdot P(r|h,t;\Theta)
\end{equation}
in which way the neighbor context and the path context of a triple are incorporated.


\subsection{Model Learning}
By feasible approximation, the score function is transformed to Eq.~\eqref{decomposition_approx}, each part is represented in softmax form as Eq.~\eqref{P_h}, Eq.~\eqref{P_t} and Eq.~\eqref{P_r}. However, it is impractical to compute these softmax functions directly because of high computational overhead. Hence, we adopt negative sampling, which is proposed in \cite{DBLP:conf/nips/MikolovSCCD13} to approximate full softmax function efficiently, to approximate softmax functions in our model. Taking $P(h|C_N(h);\Theta)$ in Eq.\eqref{P_h} as an example, it is approximated via negative sampling as follows:
\begin{equation}\label{approximation}
  %\begin{split}
    P(h|C_N(h);\Theta) \approx \sigma(f_1(h, C_N(h))) \cdot \prod_{(h',r,t) \in \mathcal{K}'}^{n} \sigma(f_1(h', C_N(h)))
  %\end{split}
\end{equation}
where $\mathcal{K}' = \{h', r, t\}$ is the corrupted triples by replacing the head entity with an arbitrary entity, $n$ is the number of negative samples and $\sigma(\cdot)$ is the logistic function. $P(t|C_P(h,t),h;\Theta)$ in Eq.~\eqref{P_t} and $P(r|h,t;\Theta)$ in Eq.~\eqref{P_r} are approximated likewise.

In real data sets, the size of neighbor context and path context may be very large, which is computationally expensive for model learning. For this reason, we sample from neighbor context and path context to make triple context tractable. Specifically, we set a threshold $n_N$ for neighbor context and $n_P$ for path context; if the size of the original context exceeds the threshold, we sample a subset, size of which is the threshold, for model learning. Moreover, the length of relation path is constrained to 2 and 3 in our model.

For computational convenience, the joint probability in Eq.~\eqref{joint_prob} is transformed to a negative logarithmic loss function, which can be optimized by \emph{stochastic gradient descent} (SGD).

\begin{comment}
the objective function is formulated as follows:
\begin{equation}\label{cost_function}
  \mathcal{L}(\mathcal{K}) = \sum_{(h,r,t)\in\mathcal{K}}
\end{equation}

For computational convenience, the original joint probability in Eq.~\eqref{joint_prob} is transformed into the negative logarithmic form, which can be optimized by stochastic gradient descent (SGD). The cost function of a triple $(h,r,t)$ is formulated as:
\begin{equation}\label{score_function}
*
\end{equation}
\end{comment}




\section{Experiments}
\begin{table*} %\small
  \centering
  \caption{Results on FB15k by relation category}
  \label{table_results_by_relation_category}
  \begin{tabular}{c|cccc|cccc}
    \hline
    Task               & \multicolumn{4}{c|}{Predicting head(\textsc{Hits}@10(\%))} & \multicolumn{4}{c}{Predicting tail(\textsc{Hits}@10(\%))} \\
    \hline
    Relation Category  & 1-To-1        & 1-To-N        & N-To-1        & N-To-N        & 1-To-1        & 1-To-N        & N-To-1        & N-To-N \\
    \hline
%    SE                 & 35.6          & 62.6          & 17.2          & 37.5          & 34.9          & 14.6          & 68.3          & 41.3   \\
%    SME(linear)        & 35.1          & 53.7          & 19.0          & 40.3          & 32.7          & 14.9          & 61.6          & 43.3   \\
%    SME(bilinear)      & 30.9          & 69.6          & 19.9          & 38.6          & 28.2          & 13.1          & 76.0          & 41.8   \\
    TransE             & 43.7          & 65.7          & 18.2          & 47.2          & 43.7          & 19.7          & 66.7          & 50.0   \\
    TransH (unif)      & 66.7          & 81.7          & 30.2          & 57.4          & 63.7          & 30.1          & 83.2          & 60.8   \\
    TransH (bern)      & 66.8          & 87.6          & 28.7          & 64.5          & 65.5          & 39.8          & 83.3          & 67.2   \\
    TransR (unif)      & 76.9          & 77.9          & 38.1          & 66.9          & 76.2          & 38.4          & 76.2          & 69.1   \\
    TransR (bern)      & 78.8          & \textbf{89.2} & 34.1          & 69.2          & 79.2          & 37.4          & \textbf{90.4} & 72.1   \\
    CTransR (unif)     & 78.6          & 77.8          & 36.4          & 68.0          & 77.4          & 37.8          & 78.0          & 70.3   \\
    CTransR (bern)     & \textbf{81.5} & 89.0          & 34.7          & 71.2          & \textbf{80.8} & 38.6          & 90.1          & 73.8   \\
    \hline
    TCE                & 71.0          & 60.3          & \textbf{83.9} & \textbf{81.9} & 70.3          & \textbf{89.9} & 76.0          & \textbf{89.2}   \\
    \hline
\end{tabular}
\end{table*}


\subsection{Experimental Setup}

\textbf{\indent Data Set.}
We use a widely-used benchmark data set FB15k~\cite{BordesUGWY13} for evaluation, which is extracted from Freebase. It has \fnum{592213} triples with \fnum{14951} entities and \fnum{1345} relationships. It is further divided into three parts for model training, tuning and evaluation.

\textbf{Evaluation protocol.}
Following the same protocol used in \cite{BordesUGWY13}, we use \textit{Mean Rank} and \textit{Hits@10} as evaluation protocals of our model. For each test triple $(h,r,t)$, we replace $h$/$t$ with each entity in $\mathcal{E}$ to generate \emph{corrupted triples} and calculate the scores of each triple using the score function. After ranking the scores in descending order, we then get the rank of the correct entity. \textit{Mean Rank} is the mean of all the predicted ranks, and \textit{Hits@10} denotes the proportion of correct entities ranked in the top 10. Note that, a corrupted triple ranking above a test triple could be valid, which should not be counted as an error. To eliminate the effects of such condition, corrupted triples that already exist in the KG are filtered before ranking. In this case, the setting of evaluation is called "Filter", while the original one is called "Raw". A lower Mean Rank and a higher Hits@10 imply the better performance of a model.

\textbf{Baselines.}
We use a few outstanding models in recent years as baselines and compare our model with them, including TransE~\cite{BordesUGWY13}, TransH~\cite{WangZFC14}, TransR~\cite{LinLSLZ15} , CTransR~\cite{LinLSLZ15}, PTransE~\cite{LinLLSRL15} and GAKE~\cite{FengHYZ16}.

\begin{table} %\small
  \caption{Link prediction results}
  \label{table_link_prediction_results}
  \begin{tabular}{c|cc|cc}
    \hline
    \multirow{2}{*}{Metric}  & \multicolumn{2}{c|}{Mean Rank} & \multicolumn{2}{c}{\textsc{Hits}@10(\%)} \\
                             & Raw          & Filter          & Raw           & Filter          \\
    \hline
%    RESCAL                   & 828          & 683             & 28.4          & 44.1            \\
%    SE                       & 273          & 162             & 28.8          & 39.8            \\
%    SME(linear)              & 274          & 154             & 30.7          & 40.8            \\
%    SME(bilinear)            & 284          & 158             & 31.3          & 41.3            \\
%    LFM                      & 283          & 164             & 26.0          & 33.1            \\
    TransE                   & 243          & 125             & 34.9          & 47.1            \\
    TransH (unif)            & 211          & 84              & 42.5          & 58.5            \\
    TransH (bern)            & 212          & 87              & 45.7          & 64.4            \\
    TransR (unif)            & 226          & 78              & 43.8          & 65.5            \\
    TransR (bern)            & 198          & 77              & 48.2          & 68.7            \\
    CTransR (unif)           & 233          & 82              & 44.0          & 66.3            \\
    CTransR (bern)           & 199          & 75              & 48.4          & 70.2            \\
    PTransE                  & 207          & 58              & 51.4          & \textbf{84.6}   \\
    GAKE                     & 228          & 119             & 44.5          & 64.8            \\
    \hline
    TCE                      & \textbf{110} & \textbf{25}     & \textbf{55.3} & 83.1            \\
    \hline
  \end{tabular}
\end{table}

\textbf{Implementation.}
We construct the knowledge graph using Apache TinkerPop\footnote{\url{http://tinkerpop.apache.org/}}, an open source graph computing framework. In a few cases, the reverse relation, an edge labeled $r^{-1}$ from $t$ to $h$ for the triple $(h,r,t)$, would be useful when representing some patterns in the graph. For instance, the relation path $a \xrightarrow{motherOf} b \xleftarrow{fatherOf} c$, i.e., $(a, motherOf, b)$ and $(c, fatherOf, b)$, indicates a potential relation $marriedTo$ between $a$ and $c$. Therefore, we add reverse relation of each relation into KG. Specifically, for each edge labeled $r$ from $h$ to $t$ in the graph, we add another edge labeled $r^{-1}$ from $t$ to $h$. In addition, the thresholds of neighbor context and path context are both set to 10.

%For neighbor context generation, it's expensive to consider all the neighbors of each entity in the graph for the reason that there are some entities connecting with a large amount of other entities, which would lead to a huge size of neighbor context. We use sampling to reduce the size of neighbor context. For those entity whose neighbor context size is larger than a fixed size $n$, we sample $n$ neighbors randomly from it's neighbor context. Similarly, a large number of paths between a pair of entities would result in high computational complexity. To solve the problem, firstly, we limit the length of paths by 2-step and 3-step, then, we use random walk to sample $m$ paths between a pair of entities. In our experiment, $n$ and $m$ are all set as 10. Note that for some pairs of entities, there may be no 2 or 3 step relation paths. In such case, we suppose that the relatedness between those pairs of entities are relatively low and the values of $f_2()$ in Eq.~\eqref{f2} are set as -100.


We use mini-batch SGD to train our model. We choose the learning rate $\alpha$ of SGD among $\{0.1, 0.01, 0.001\}$, the dimension of embeddings $k$ among $\{50, 75, 100\}$, the batch size $B$ among $\{120, 480,$ $ 960, 1920, 4800\}$. The best parameters are determined by the performance on valid set. The optimal parameters are $\alpha=0.001$, $k=50$ and $B=4800$.

\subsection{Link Prediction}
Link prediction~\cite{BordesUGWY13} is to predict the missing head or tail entity in a given triple based on training triples. Metrics \textit{Mean Rank} and \textit{Hits@10} are used to measure the performance of our model.


We collected the result of link prediction in Table~\ref{table_link_prediction_results}. From the results we can see that, our model outperforms other baselines on most of the metrics significantly and consistently, while slightly worse than PTransE on \textsc{Hits}@10. The result implies that triple contexts do improve the performance on link prediction. Although using similar types of contexts in the graph, GAKE's performance is inferior to our model, which shows the superiority of our framework. Note that the experimental results of \textsc{HolE} are absent here, for it uses a different metric, MRR (Mean Reciprocal rank), instead of \textit{Mean rank} for evaluation. But according to \textit{Hits@10} reported in~\cite{NickelRP16}, the results of our model are better than \textsc{HolE}.



In Table~\ref{table_results_by_relation_category}, we show separate evaluation results by category of relationships on FB15k. We can see that TCE brings promising improvements on modeling complex relations, such as predicting tail of 1-To-N relations, predicting head of N-To-1 relations and N-To-N relations. Specifically, TCE behaves well when predicting the "N" side of 1-To-N and N-To-1 relations, indicating that valid triples have higher scores than invalid triples in general. In some other simpler scenarios, such as 1-To-1 relations and predicting the "1" side of 1-To-N and N-To-1 relations, the performance of TCE is still acceptable although not so good as some other baselines, such as TransH and TransR. The results suggest that the incorporation of triple context is helpful when handling complex relations, at the cost of precision in modeling simple relations, which seems complementary to some other baselines.



\section{Conclusion and Future Work}
In this paper, we proposed TCE, a KG embedding model which is able to take advantages of the triple context in the graph. By defining two kinds of context of a triple and representing them in a unified framework, our model can learn embeddings that are aware of their context. We evaluate our model on link prediction and the experimental results show significant improvements over the major baselines.

In the future, we will research on the following aspects: (1) Conduct experiments on more data sets and tasks to validate our model. (2) Current results show complementarity to some other methods such as TransH, TransR. We would think about a combination of those methods and our model.
%compare the performance of TCE on different data sets, and analyze the characters of data sets that will make the most of triple context in the graph
% Among all the contexts of a triple in the KG, some are more important than others and may contain more useful patterns. In future, we would explore a method to extract \textit{key contexts} instead of sampling randomly.


\bibliographystyle{abbrv}
\bibliography{reference}

\end{document}
